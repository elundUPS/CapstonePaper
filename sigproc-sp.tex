% THIS IS SIGPROC-SP.TEX - VERSION 3.1
% WORKS WITH V3.2SP OF ACM_PROC_ARTICLE-SP.CLS
% APRIL 2009
%
% It is an example file showing how to use the 'acm_proc_article-sp.cls' V3.2SP
% LaTeX2e document class file for Conference Proceedings submissions.
% ----------------------------------------------------------------------------------------------------------------
% This .tex file (and associated .cls V3.2SP) *DOES NOT* produce:
%       1) The Permission Statement
%       2) The Conference (location) Info information
%       3) The Copyright Line with ACM data
%       4) Page numbering
% ---------------------------------------------------------------------------------------------------------------
% It is an example which *does* use the .bib file (from which the .bbl file
% is produced).
% REMEMBER HOWEVER: After having produced the .bbl file,
% and prior to final submission,
% you need to 'insert'  your .bbl file into your source .tex file so as to provide
% ONE 'self-contained' source file.
%
% Questions regarding SIGS should be sent to
% Adrienne Griscti ---> griscti@acm.org
%
% Questions/suggestions regarding the guidelines, .tex and .cls files, etc. to
% Gerald Murray ---> murray@hq.acm.org
%
% For tracking purposes - this is V3.1SP - APRIL 2009

\documentclass{acm_proc_article-sp}

\begin{document}

\title{Computer Science Capstone Project {\ttlit Fit n' Lift}\titlenote{Created for the Computer Science Capstone course (CSCI440) at the University of Puget Sound}}

%
% You need the command \numberofauthors to handle the 'placement
% and alignment' of the authors beneath the title.
%
% For aesthetic reasons, we recommend 'three authors at a time'
% i.e. three 'name/affiliation blocks' be placed beneath the title.
%
% NOTE: You are NOT restricted in how many 'rows' of
% "name/affiliations" may appear. We just ask that you restrict
% the number of 'columns' to three.
%
% Because of the available 'opening page real-estate'
% we ask you to refrain from putting more than six authors
% (two rows with three columns) beneath the article title.
% More than six makes the first-page appear very cluttered indeed.
%
% Use the \alignauthor commands to handle the names
% and affiliations for an 'aesthetic maximum' of six authors.
% Add names, affiliations, addresses for
% the seventh etc. author(s) as the argument for the
% \additionalauthors command.
% These 'additional authors' will be output/set for you
% without further effort on your part as the last section in
% the body of your article BEFORE References or any Appendices.

\numberofauthors{2} %  in this sample file, there are a *total*
% of EIGHT authors. SIX appear on the 'first-page' (for formatting
% reasons) and the remaining two appear in the \additionalauthors section.
%
\author{
% You can go ahead and credit any number of authors here,
% e.g. one 'row of three' or two rows (consisting of one row of three
% and a second row of one, two or three).
%
% The command \alignauthor (no curly braces needed) should
% precede each author name, affiliation/snail-mail address and
% e-mail address. Additionally, tag each line of
% affiliation/address with \affaddr, and tag the
% e-mail address with \email.
%
% 1st. author
\alignauthor
Eric Lund\\
       \affaddr{University of Puget Sound}\\
       \affaddr{1500 N. Warner St.}\\
       \affaddr{Tacoma, WA 98416}\\
       \email{elund@pugetsound.edu}
% 2nd. author
\alignauthor
Lauren Swanson\\
       \affaddr{University of Puget Sound}\\
       \affaddr{1500 N. Warner St.}\\
       \affaddr{Tacoma, WA 98416}\\
       \email{lswanson@pugetsound.edu}
       \and 
% 3rd. author
\alignauthor Todd Detweiler\\
       \affaddr{University of Puget Sound}\\
       \affaddr{1500 N. Warner St.}\\
       \affaddr{Tacoma, WA 98416}\\
       \email{tdetweiler@pugetsound.edu}
 % use '\and' if you need 'another row' of author names
% 4th. author
\alignauthor Gregory Finch\\
       \affaddr{University of Puget Sound}\\
       \affaddr{1500 N. Warner St.}\\
       \affaddr{Tacoma, WA 98416}\\
       \email{gfinch@pugetsound.edu}
}
% There's nothing stopping you putting the seventh, eighth, etc.
% author on the opening page (as the 'third row') but we ask,
% for aesthetic reasons that you place these 'additional authors'
% in the \additional authors block, viz.
\additionalauthors{Additional authors: John Smith (The Th{\o}rv{\"a}ld Group,
email: {\texttt{jsmith@affiliation.org}}) and Julius P.~Kumquat
(The Kumquat Consortium, email: {\texttt{jpkumquat@consortium.net}}).}
\date{30 July 1999}
% Just remember to make sure that the TOTAL number of authors
% is the number that will appear on the first page PLUS the
% number that will appear in the \additionalauthors section.

\maketitle
\begin{abstract}
A short (max 150 words) summary of the paper. It's usually easier to write this last, once the structure of the paper has taken shape.A short (max 150 words) summary of the paper. It's usually easier to write this last, once the structure of the paper has taken shape.A short (max 150 words) summary of the paper. It's usually easier to write this last, once the structure of the paper has taken shape.A short (max 150 words) summary of the paper. It's usually easier to write this last, once the structure of the paper has taken shape.A short (max 150 words) summary of the paper. It's usually easier to write this last, once the structure of the paper has taken shape.
\end{abstract}

% A category with the (minimum) three required fields
%\category{H.4}{Information Systems Applications}{Miscellaneous}
%A category including the fourth, optional field follows...
%\category{D.2.8}{Software Engineering}{Metrics}[complexity measures, performance measures]

%\terms{Theory}

\keywords{Social Media, Mobile Development, Capstone} % NOT required for Proceedings

\section{Introduction}
 Explain the context and motivation for your project. What domain are you working in? What is the subject of your project? Why is this a problem worth solving? Include a very brief description of your approach. With some luck, you can probably steal most of this from your project proposal. Explain the context and motivation for your project. What domain are you working in? What is the subject of your project? Why is this a problem worth solving? Include a very brief description of your approach. With some luck, you can probably steal most of this from your project proposal. Explain the context and motivation for your project. What domain are you working in? What is the subject of your project? Why is this a problem worth solving? Include a very brief description of your approach. With some luck, you can probably steal most of this from your project proposal. Explain the context and motivation for your project. What domain are you working in? What is the subject of your project? Why is this a problem worth solving? Include a very brief description of your approach. With some luck, you can probably steal most of this from your project proposal.

\section{{\secit Background and Related Work:}}
What other work has been done before? Do similar systems or research exist? How does your project fit into a greater context, perhaps building on what has come before? Be sure to reference other work properly, and include those references at the end of the paper. What other work has been done before? Do similar systems or research exist? How does your project fit into a greater context, perhaps building on what has come before? Be sure to reference other work properly, and include those references at the end of the paper. What other work has been done before? Do similar systems or research exist? How does your project fit into a greater context, perhaps building on what has come before? Be sure to reference other work properly, and include those references at the end of the paper. What other work has been done before? Do similar systems or research exist? How does your project fit into a greater context, perhaps building on what has come before? Be sure to reference other work properly, and include those references at the end of the paper. What other work has been done before? Do similar systems or research exist? How does your project fit into a greater context, perhaps building on what has come before? Be sure to reference other work properly, and include those references at the end of the paper.


\section{Implementation / Method}
\subsection{Website}
If you did an implementation project, describe its architecture and how it works. For a research project, explain your research method. Feel free to document failed attempts as well as the final path taken. This is likely to be the largest section in the writeup of an implementation project. Just the facts please: this section should not make claims about how well your system works or how it compares to other efforts. That material goes in the next two sections. If you did an implementation project, describe its architecture and how it works. For a research project, explain your research method. Feel free to document failed attempts as well as the final path taken. This is likely to be the largest section in the writeup of an implementation project. Just the facts please: this section should not make claims about how well your system works or how it compares to other efforts. That material goes in the next two sections. If you did an implementation project, describe its architecture and how it works. For a research project, explain your research method. Feel free to document failed attempts as well as the final path taken. This is likely to be the largest section in the writeup of an implementation project. Just the facts please: this section should not make claims about how well your system works or how it compares to other efforts. That material goes in the next two sections. If you did an implementation project, describe its architecture and how it works. For a research project, explain your research method. Feel free to document failed attempts as well as the final path taken. This is likely to be the largest section in the writeup of an implementation project. Just the facts please: this section should not make claims about how well your system works or how it compares to other efforts. That material goes in the next two sections.

\subsection{iOS}
If you did an implementation project, describe its architecture and how it works. For a research project, explain your research method. Feel free to document failed attempts as well as the final path taken. This is likely to be the largest section in the writeup of an implementation project. Just the facts please: this section should not make claims about how well your system works or how it compares to other efforts. That material goes in the next two sections. If you did an implementation project, describe its architecture and how it works. For a research project, explain your research method. Feel free to document failed attempts as well as the final path taken. This is likely to be the largest section in the writeup of an implementation project. Just the facts please: this section should not make claims about how well your system works or how it compares to other efforts. That material goes in the next two sections. If you did an implementation project, describe its architecture and how it works. For a research project, explain your research method. Feel free to document failed attempts as well as the final path taken. This is likely to be the largest section in the writeup of an implementation project. Just the facts please: this section should not make claims about how well your system works or how it compares to other efforts. That material goes in the next two sections. If you did an implementation project, describe its architecture and how it works. For a research project, explain your research method. Feel free to document failed attempts as well as the final path taken. This is likely to be the largest section in the writeup of an implementation project. Just the facts please: this section should not make claims about how well your system works or how it compares to other efforts. That material goes in the next two sections.

%ACKNOWLEDGMENTS are optional
\section{Results and Analysis}
\subsection{Website}
How well does your system work, or what did you figure out? In an implementation project, you might report on either a short user study explaining the system in use, or give the results of tests demonstrating the effectiveness of your implementation. For research projects, present and analyze data that supports your arguments. It wouldn't be surprising for this to be the largest section in a research project writeup. How well does your system work, or what did you figure out? In an implementation project, you might report on either a short user study explaining the system in use, or give the results of tests demonstrating the effectiveness of your implementation. For research projects, present and analyze data that supports your arguments. It wouldn't be surprising for this to be the largest section in a research project writeup. How well does your system work, or what did you figure out? In an implementation project, you might report on either a short user study explaining the system in use, or give the results of tests demonstrating the effectiveness of your implementation. For research projects, present and analyze data that supports your arguments. It wouldn't be surprising for this to be the largest section in a research project writeup.

\subsection{iOS}
If you did an implementation project, describe its architecture and how it works. For a research project, explain your research method. Feel free to document failed attempts as well as the final path taken. This is likely to be the largest section in the writeup of an implementation project. Just the facts please: this section should not make claims about how well your system works or how it compares to other efforts. That material goes in the next two sections. If you did an implementation project, describe its architecture and how it works. For a research project, explain your research method. Feel free to document failed attempts as well as the final path taken. This is likely to be the largest section in the writeup of an implementation project. Just the facts please: this section should not make claims about how well your system works or how it compares to other efforts. That material goes in the next two sections. If you did an implementation project, describe its architecture and how it works. For a research project, explain your research method. Feel free to document failed attempts as well as the final path taken. This is likely to be the largest section in the writeup of an implementation project. Just the facts please: this section should not make claims about how well your system works or how it compares to other efforts. That material goes in the next two sections. If you did an implementation project, describe its architecture and how it works. For a research project, explain your research method. Feel free to document failed attempts as well as the final path taken. This is likely to be the largest section in the writeup of an implementation project. Just the facts please: this section should not make claims about how well your system works or how it compares to other efforts. That material goes in the next two sections.


\section{Discussion}
\subsection{Website}
The previous two sections described your work in detail, and the related work section talked about what others have done. This section should put your results in context — how does your contribution support, refute, or extend previous work? What is the significance of your project? Opinions are fine here, as long as you make a coherent argument to back them up. (E.g. "We feel that this system offers a superior experience for the user since it crashes less often than Microsoft Blurb.") Your discussion will be stronger if you give an honest comparison of your project to existing work. (E.g. "It is true that Microsoft Blurb offers better security than our system, but security was not our primary emphasis for this proof-of-concept implementation, and could be added at later date.") The previous two sections described your work in detail, and the related work section talked about what others have done. This section should put your results in context — how does your contribution support, refute, or extend previous work? What is the significance of your project? Opinions are fine here, as long as you make a coherent argument to back them up. (E.g. "We feel that this system offers a superior experience for the user since it crashes less often than Microsoft Blurb.") Your discussion will be stronger if you give an honest comparison of your project to existing work. (E.g. "It is true that Microsoft Blurb offers better security than our system, but security was not our primary emphasis for this proof-of-concept implementation, and could be added at later date.")

\subsection{iOS}
The previous two sections described your work in detail, and the related work section talked about what others have done. This section should put your results in context — how does your contribution support, refute, or extend previous work? What is the significance of your project? Opinions are fine here, as long as you make a coherent argument to back them up. (E.g. "We feel that this system offers a superior experience for the user since it crashes less often than Microsoft Blurb.") Your discussion will be stronger if you give an honest comparison of your project to existing work. (E.g. "It is true that Microsoft Blurb offers better security than our system, but security was not our primary emphasis for this proof-of-concept implementation, and could be added at later date.") The previous two sections described your work in detail, and the related work section talked about what others have done. This section should put your results in context — how does your contribution support, refute, or extend previous work? What is the significance of your project? Opinions are fine here, as long as you make a coherent argument to back them up. (E.g. "We feel that this system offers a superior experience for the user since it crashes less often than Microsoft Blurb.") Your discussion will be stronger if you give an honest comparison of your project to existing work. (E.g. "It is true that Microsoft Blurb offers better security than our system, but security was not our primary emphasis for this proof-of-concept implementation, and could be added at later date.")


\section{Future Work}
\subsection{Website}
What are the next steps, either for you or for people who might build on your work?What are the next steps, either for you or for people who might build on your work?What are the next steps, either for you or for people who might build on your work?What are the next steps, either for you or for people who might build on your work?What are the next steps, either for you or for people who might build on your work?What are the next steps, either for you or for people who might build on your work?What are the next steps, either for you or for people who might build on your work?What are the next steps, either for you or for people who might build on your work?What are the next steps, either for you or for people who might build on your work?What are the next steps, either for you or for people who might build on your work?What are the next steps, either for you or for people who might build on your work?What are the next steps, either for you or for people who might build on your work?What are the next steps, either for you or for people who might build on your work?

\subsection{iOS}
What are the next steps, either for you or for people who might build on your work?What are the next steps, either for you or for people who might build on your work?What are the next steps, either for you or for people who might build on your work?What are the next steps, either for you or for people who might build on your work?What are the next steps, either for you or for people who might build on your work?What are the next steps, either for you or for people who might build on your work?What are the next steps, either for you or for people who might build on your work?What are the next steps, either for you or for people who might build on your work?What are the next steps, either for you or for people who might build on your work?What are the next steps, either for you or for people who might build on your work?What are the next steps, either for you or for people who might build on your work?What are the next steps, either for you or for people who might build on your work?


\section{Conclusion}
Include some concluding remarks. Summarize and reiterate what you see as the key points from your writeup. Basically you're using the old "tell them what you're going to tell them, tell them, and then tell them what you told them" structure in the hopes that your point really got through.Include some concluding remarks. Summarize and reiterate what you see as the key points from your writeup. Basically you're using the old "tell them what you're going to tell them, tell them, and then tell them what you told them" structure in the hopes that your point really got through.Include some concluding remarks. Summarize and reiterate what you see as the key points from your writeup. Basically you're using the old "tell them what you're going to tell them, tell them, and then tell them what you told them" structure in the hopes that your point really got through.Include some concluding remarks. Summarize and reiterate what you see as the key points from your writeup. Basically you're using the old "tell them what you're going to tell them, tell them, and then tell them what you told them" structure in the hopes that your point really got through.


\section{Acknowledgments}
This section is used to acknowledge the people or organizations that provided significant help or inspiration. It might not be relevant to your writeup. In the real world you'd thank the funding agencies that supported your work, the reviewers who gave feedback on drafts of the paper, and any other contributors. If you did an implementation project, you might consider acknowledging the authors of any packages or libraries you used, etc.This section is used to acknowledge the people or organizations that provided significant help or inspiration. It might not be relevant to your writeup. In the real world you'd thank the funding agencies that supported your work, the reviewers who gave feedback on drafts of the paper, and any other contributors. If you did an implementation project, you might consider acknowledging the authors of any packages or libraries you used, etc.

\section{References}
A properly formatted list of references. These can be construed fairly broadly in computer science — in addition to journal articles, books, and conference proceedings, papers often reference web pages, tutorials, etc. Note that this is different from a Bibliography, which lists things that you've read but that are not necessarily referenced in your paper. All of the entries in this section should be referenced at least once in the body of your paper.

\section{Appendicies}
You might consider adding an appendix (or several) if you have tables of data, long listings of source code, or other supporting information that's too long to insert into the body of the paper, but that's relevant to your project. These will not be counted against the 8-10 page limit.
You might consider adding an appendix (or several) if you have tables of data, long listings of source code, or other supporting information that's too long to insert into the body of the paper, but that's relevant to your project. These will not be counted against the 8-10 page limit.
You might consider adding an appendix (or several) if you have tables of data, long listings of source code, or other supporting information that's too long to insert into the body of the paper, but that's relevant to your project. These will not be counted against the 8-10 page limit.

%
% The following two commands are all you need in the
% initial runs of your .tex file to
% produce the bibliography for the citations in your paper.
%%\bibliographystyle{abbrv}
%%\bibliography{sigproc}  % sigproc.bib is the name of the Bibliography in this case
% You must have a proper ".bib" file
%  and remember to run:
% latex bibtex latex latex
% to resolve all references
%
% ACM needs 'a single self-contained file'!
%
%APPENDICES are optional
%\balancecolumns
%\appendix
%%Appendix A
%\section{Headings in Appendices}
%The rules about hierarchical headings discussed above for
%the body of the article are different in the appendices.
%In the \textbf{appendix} environment, the command
%\textbf{section} is used to
%indicate the start of each Appendix, with alphabetic order
%designation (i.e. the first is A, the second B, etc.) and
%a title (if you include one).  So, if you need
%hierarchical structure
%\textit{within} an Appendix, start with \textbf{subsection} as the
%highest level. Here is an outline of the body of this
%document in Appendix-appropriate form:
%\subsection{Introduction}
%\subsection{The Body of the Paper}
%\subsubsection{Type Changes and  Special Characters}
%\subsubsection{Math Equations}
%\paragraph{Inline (In-text) Equations}
%\paragraph{Display Equations}
%\subsubsection{Citations}
%\subsubsection{Tables}
%\subsubsection{Figures}
%\subsubsection{Theorem-like Constructs}
%\subsubsection*{A Caveat for the \TeX\ Expert}
%\subsection{Conclusions}
%\subsection{Acknowledgments}
%\subsection{Additional Authors}
%This section is inserted by \LaTeX; you do not insert it.
%You just add the names and information in the
%\texttt{{\char'134}additionalauthors} command at the start
%of the document.
%\subsection{References}
%Generated by bibtex from your ~.bib file.  Run latex,
%then bibtex, then latex twice (to resolve references)
%to create the ~.bbl file.  Insert that ~.bbl file into
%the .tex source file and comment out
%the command \texttt{{\char'134}thebibliography}.
%% This next section command marks the start of
%% Appendix B, and does not continue the present hierarchy
%\section{More Help for the Hardy}
%The acm\_proc\_article-sp document class file itself is chock-full of succinct
%and helpful comments.  If you consider yourself a moderately
%experienced to expert user of \LaTeX, you may find reading
%it useful but please remember not to change it.
%\balancecolumns
%% That's all folks!
\end{document}
